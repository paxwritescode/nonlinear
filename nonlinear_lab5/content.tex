\section{Постановка задачи}
Необходимо на основе метода Рунге-Кутты построить двусторонние оценки решения следующей задачи Коши:
$$
\begin{cases}
	y'(x) = y cos(tx) \\
	y(0) = 1
\end{cases}
$$
Точным решением этой задачи является функция $y(x) = e^{\frac{sin(tx)}{t}}$.
\begin{figure}[H]
	\centering
	\begin{subfigure}{0.45\textwidth}
		\centering
		\includegraphics[width=\linewidth]{img/exact/exact1.png}
		\caption{$t = 1$}
	\end{subfigure}
	\hfill
	\begin{subfigure}{0.45\textwidth}
		\centering
		\includegraphics[width=\linewidth]{img/exact/exact2.png}
		\caption{$t = 10$}
	\end{subfigure}
	\caption{Точное решение задачи Коши при различных значениях параметра}
	\label{fig:two_graphs}
\end{figure}

\section{Цель работы}
Требуется на примере данной задачи исследовать эффективность двустороннего метода и сравнить его с методом Эйлера. Сравнение методоы нужно провести при различных значениях параметра $t$, рассмотрев сетки с шагом $h = 0.01, \: h = 0.005 \: \text{и} \: h = 0.0025$.

\section{Алгоритмы методов}
Поставлена задача Коши:
$$
\begin{cases}
	y'(x) = f(x, y) \\
	y(a) = y_0
\end{cases}
$$
Необходимо вычислить значения функции $y$ на множестве узлов $\{x_i\}_{i = 0}^n$, где $x_0 = a, \: x_n = b \: \text{и} \: x_i < x_{i + 1}.$
В данной лабораторной работе рассматриваются вычислительные схемы, относящиеся к группе методов Рунге-Кутты. Они основаны на использовании для перехода от узла $x_i$ к узлу $x_{i+1}$ определенного количества слагаемых из разложения функции $y$ в ряд Тейлора:
\[
y(x + h) = y(x) + h y'(x) + \frac{h^2}{2} y''(x) + \frac{h^3}{6} y'''(x) + \dots,
\]

где $x = x_i$, $h = x_{i+1} - x_i$.
\subsection{Метод Эйлера}
Метод Эйлера является простейшим из группы методов Рунге-Кутты. Его вычислительная схема выглядит следующим образом:
$$y(x + h) \approx y(x) + h f(x, y)$$
\subsection{Двусторонний метод Рунге-Кутты}
Двусторонний метод использует две независимые аппроксимации решения с различными коэффициентами, что позволяет получить верхнюю и нижнюю границы для значения $y(x)$ на каждом шаге.\\
\underline{Первый шаг метода: двусторонние оценки для $x_1$.}\\
Вычислим три промежуточных значения:
$$K_1 = f(x, y)$$
$$K_2 = f(x + h \alpha_2, y + h \beta_{21}K_1)$$
$$K_3 = f(x + h \alpha_3, y + h(\beta_{31}K_1 + \beta_{32} K_2)),$$
где $\alpha_2, \: \alpha_3, \: \beta_{21}, \: \beta_{31}, \beta_{32}$ - коэффициенты метода, обеспечивающие его порядок точности.
На основе этих значений строятся две различные оценки для $y(x_1) = y(x_0 + h)$:
\[
y^{(1)}(x + h) = y(x) + h (p_1 K_1 + p_2 K_2 + p_3 K_3)
\]
\[
y^{(2)}(x + h) = y(x) + h (\tilde{p}_1 K_1 + \tilde{p}_2 K_2 + \tilde{p}_3 K_3)
\]
Коэффициенты $p_1, \: p_2, \: p_3 \; \text{и} \; \tilde{p}_1, \tilde{p}_2, \: \tilde{p}_3 $ подобраны так, чтобы одна из формул переоценивала точное решение, а другая - недооценивала его.\\
В результате на первом узле $x_1$ получаем две границы для точного значения:
$$min(y^{(1)}, y^{(2)}) \leqslant y(x_1) \leqslant max(y^{(1)}, y^{(2)})$$
Таким образом, на первом узле мы получаем 2 приближённых значения.\\
\underline{Построение двусторонней оценки на i-м узле}\\
На каждом шаге строятся два новых значения на основе каждого из полученных на предыдущем шаге. Таким образом, на каждом следующем шаге количество аппроксимаций удваивается, следовательно, на i-м узле получается $2^i$ значений. Среди них выбираются максимальное и минимальное. Эти границы обеспечивают интервал, в который попадёт точка решения.

\section{Результаты}
\subsection{Визуализация двусторонних оценок}
Построим графики решения ДУ двусторонним методом Рунге-Кутты. Для наглядности выберем малое число узлов.
\begin{figure}[H]
	\centering
	\begin{subfigure}{0.45\textwidth}
		\centering
		\includegraphics[width=\linewidth]{img/visualization/TwoSided3nodest1}
		\caption{$t = 1$}
	\end{subfigure}
	\hfill
	\begin{subfigure}{0.45\textwidth}
		\centering
		\includegraphics[width=\linewidth]{img/visualization/TwoSided3nodest2}
		\caption{$t = 10$}
	\end{subfigure}
	\caption{Двусторонние оценки при 5 узлах и различных значениях параметра}
	\label{fig:two_graphs}
\end{figure}

\begin{figure}[H]
	\centering
	\begin{subfigure}{0.45\textwidth}
		\centering
		\includegraphics[width=\linewidth]{img/visualization/TwoSided5nodest1}
		\caption{$t = 1$}
	\end{subfigure}
	\hfill
	\begin{subfigure}{0.45\textwidth}
		\centering
		\includegraphics[width=\linewidth]{img/visualization/TwoSided5nodest2}
		\caption{$t = 10$}
	\end{subfigure}
	\caption{Двусторонние оценки при 5 узлах и различных значениях параметра}
	\label{fig:two_graphs}
\end{figure}

\begin{figure}[H]
	\centering
	\begin{subfigure}{0.45\textwidth}
		\centering
		\includegraphics[width=\linewidth]{img/visualization/TwoSided11nodest1}
		\caption{$t = 1$}
	\end{subfigure}
	\hfill
	\begin{subfigure}{0.45\textwidth}
		\centering
		\includegraphics[width=\linewidth]{img/visualization/TwoSided11nodest2}
		\caption{$t = 10$}
	\end{subfigure}
	\caption{Двусторонние оценки при 11 узлах и различных значениях параметра}
	\label{fig:two_graphs}
\end{figure}

\subsection{Сравнение погрешностей метода Эйлера и двустороннего метода Рунге-Кутты}
\begin{figure}[H]
	\centering
	\begin{subfigure}{0.45\textwidth}
		\centering
		\includegraphics[width=\linewidth]{img/errorEulerAndDifferenceBtwEstimates/ErrEuDifRK11}
		\caption{$t = 1$}
	\end{subfigure}
	\hfill
	\begin{subfigure}{0.45\textwidth}
		\centering
		\includegraphics[width=\linewidth]{img/errorEulerAndDifferenceBtwEstimates/ErrEuDifRK12}
		\caption{$t = 10$}
	\end{subfigure}
	\caption{Двусторонние оценки при 101 узлах и различных значениях параметра}
	\label{fig:two_graphs}
\end{figure}


\begin{figure}[H]
	\centering
	\begin{subfigure}{0.45\textwidth}
		\centering
		\includegraphics[width=\linewidth]{img/errorEulerAndDifferenceBtwEstimates/ErrEuDifRK21}
		\caption{$t = 1$}
	\end{subfigure}
	\hfill
	\begin{subfigure}{0.45\textwidth}
		\centering
		\includegraphics[width=\linewidth]{img/errorEulerAndDifferenceBtwEstimates/ErrEuDifRK22}
		\caption{$t = 10$}
	\end{subfigure}
	\caption{Двусторонние оценки при 201 узлах и различных значениях параметра}
	\label{fig:two_graphs}
\end{figure}

\begin{figure}[H]
	\centering
	\begin{subfigure}{0.45\textwidth}
		\centering
		\includegraphics[width=\linewidth]{img/errorEulerAndDifferenceBtwEstimates/ErrEuDifRK31}
		\caption{$t = 1$}
	\end{subfigure}
	\hfill
	\begin{subfigure}{0.45\textwidth}
		\centering
		\includegraphics[width=\linewidth]{img/errorEulerAndDifferenceBtwEstimates/ErrEuDifRK32}
		\caption{$t = 10$}
	\end{subfigure}
	\caption{Двусторонние оценки при 401 узлах и различных значениях параметра}
	\label{fig:two_graphs}
\end{figure}


\section {Выводы}
% В ходе выполнения работы был реализован алгоритм двустороннего метода Рунге-Кутты, затем была исследована его эффективность и проведено сравнение метода Рунге-Кутты с методом Эйлера на примере двух задач Коши с известным точным решением.
% По результатам исследования можно заметить, что метод Рунге-Кутты превосходит метод Эйлера по точности. Если при решении задачи, точное решение которой имеет производную одного знака на данном отрезке, преимущество двустороннего метода Рунге-Кутты не так заметно (разница в 2-3 порядка), то на задаче, производная решения которой меняет знак, преимущество двустороннего метода Рунге-Кутты является неоспоримым, метод Эйлера к такой задаче практически неприменим из-за большой ошибки, а метод Рунге-Кутты даёт практически такие же результаты, как на задаче с производной одного знака.
% К недостаткам двустороннего метода Рунге-Кутты нужно отнести его трудоёмкость: на узле с номером $i$ требуется находить минимум и максимум среди $2^i$ значений, что требует больших вычислительных затрат.

В ходе выполнения работы был реализован алгоритм двустороннего метода Рунге-Кутты, проведено исследование его эффективности и выполнено сравнение с методом Эйлера на примере двух задач Коши с известным точным решением.

Анализ результатов показал, что метод Рунге-Кутты значительно превосходит метод Эйлера по точности. В случае задачи, точное решение которой имеет производную одного знака на заданном отрезке, преимущество двустороннего метода выражается в разнице порядка 2–3 десятичных разрядов. Однако при решении задачи, в которой производная изменяет знак, двусторонний метод Рунге-Кутты демонстрирует явное превосходство: метод Эйлера оказывается практически неприменим из-за высокой ошибки, в то время как метод Рунге-Кутты остаётся работоспособным и даёт значительно более точные результаты. При этом наблюдается небольшое снижение точности по сравнению с задачей, где производная не меняет знак.

Основным недостатком двустороннего метода Рунге-Кутты является его высокая вычислительная сложность. На узле с номером $i$i необходимо определять минимум и максимум среди $2^i$ значений, что значительно увеличивает затраты на вычисления.